\documentclass{report}
\usepackage{spverbatim}
\setlength{\parindent}{0cm}

\title{Let Us C}
\author{Manish}

\begin{document}

\maketitle

\chapter{Getting Started}

\textbf{[A] Which of the following are invalid variable names \& why?}
\begin{enumerate}
    \renewcommand{\labelenumi}{\alph{enumi}}
  \item BASICSALARY, \_basic, mindovermatter, FLOAT, hELLO  
    \subitem are valid variable names.
  \item basic-hra, \#MEAN, team’svictory, Plot\#3
    \subitem invalid variable names: contains special symbols
  \item 421, 2015\_DDay
    \subitem invalid due to: starting with numbers
  \item group., queue.
    \subitem are invalid due to the period symbol.
  \item population in 2006, over time,
    \subitem invalid due to: containing spaces
\end{enumerate}

\textbf{[B] Point out the errors, if any, in the following C statements:}
\begin{enumerate}
    \renewcommand{\labelenumi}{\alph{enumi}}
  \item no variable initialized or int is an invalid variable name.
  \item no error
  \item no error
  \item cannot have multiple variables on the left side of the assignment operator.
  \item * is not used to multiply.
  \item spaces in the variable names.
  \item no error
  \item `**' is invalid operator
  \item invalid operator
  \item implicit multipication is not allowed (skipped * operator)
  \item cannot assign value to a constant.
  \item no error
  \item string must be enclosed in ''(double quotes).
\end{enumerate}

\textbf{[C] Evaluate the following expressions and show their hierarchy.}
\begin{enumerate}
    \renewcommand{\labelenumi}{\alph{enumi}}
  \item \begin{quote}
    g = 2/2 + 2*4/2 - 2 + 2.5/3\\
    g = 1 + 8/2 - 2 + 0.83\\
    g = 1 + 4 - 2 + 0.83\\
    g = 3.83\\
  \end{quote}
  \item \begin{quote}
      on = 4*1/2 + 3/2*1 + 2 + 3.2\\
      on = 4/2 + 1*1 + 2 + 3.2\\
      on = 2 + 1 + 2 + 3.2\\
      on = 8\\
  \end{quote}
  \item \begin{quote}
      s = 4*2/4 - 6/2 + 2/3*6/2\\
      s = 8/4 - 3 + 0*6/2\\
      s = 2 - 3 + 0/2\\
      s = 2 - 3 + 0\\
      s = -1\\
  \end{quote}
  \item \begin{quote}
      s = 1/3*4/4 - 6/2 + 2/3*6/3\\
      s = 0*1 - 3 + 0*2\\
      s = 0 - 3 + 0\\
      s = -3\\
  \end{quote}
\end{enumerate}

\textbf{[D] Fill the following table for the expressions given below and then evaluate the result.}
\begin{verbatim}
Operator        Left        Right       Remark
a)  /           10          5 or 5/2/1  Left operand is unambiguous, Right is not
b)  /           3           2           Left operand is unambiguous, Right is not 
c)  +           3           4           both are unambiguous
\end{verbatim}

\textbf{[E] Convert the following equations into correspondign C statements.}
\begin{enumerate}
    \renewcommand{\labelenumi}{\alph{enumi}}
  \item \begin{quote}
      Z = (8.8*(a+b)*2/c-0.5+2*a/(q+r)) / (a+b)*(1/m))
  \end{quote}
  \item \begin{quote}
      X = (-b+(b*b)+24*a*c)/(2*a)
  \end{quote}
  \item \begin{quote}
      R = (2*v + 6.22 * (c+d))/ (g+v)
  \end{quote}
  \item \begin{quote}
      A = (7.7*b*(xy+a)/c - 0.8 + 2*b) / ( (x+a)*(1/y) )
  \end{quote}
\end{enumerate}

\textbf{[F] What would be the output of the following programs:}
\begin{enumerate}
    \renewcommand{\labelenumi}{\alph{enumi}}
  \item 0 2 0.000000 2.000000
  \item a = 0 b = -6
  \item 1
  \item \begin{quote}
      nn\\
\\
      nn\\
      nn /n/n nn\\
  \end{quote}
  \item a = ``value of a'' b = ``value of b entered by user''
  \item p = garbageValue q = garbageValue
\end{enumerate}

\textbf{[G] Pick up the correct alt for each of the following questions:}
\begin{enumerate}
    \renewcommand{\labelenumi}{\alph{enumi}}
  \item Dennis Ritchie
  \item All the above
  \item a compiler
  \item ASCII form only
  \item 1 character
  \item ASCII value of Z
  \item All of the above
  \item 32767
  \item a number
  \item 3 + a = b
  \item / or *, - or +
  \item 6.6 / a
  \item ( )
  \item Each new C instruction has to be written on a separate line.
  \item 2
  \item 0
  \item 32768
  \item 6
  \item is used first
  \item at least one digit
  \item 1 character
  \item all the above
  \item keywords can be used as variable names
  \item constants on the right side of =
  \item */+-
  \item 0.2857
\end{enumerate}

\textbf{[H] Write C programs fro the following:}
\begin{enumerate}
    \renewcommand{\labelenumi}{\alph{enumi}}
  \item \begin{quote} \begin{verbatim}
int main() {
  float basic, gross;
  scanf("%f", &basic);
  gross = basic*40/100 + basic*20/100;
  printf("gross sallary = %f\n", gross);
  return 0;
}

  \end{verbatim} \end{quote} 

  \item \begin{quote} \begin{verbatim}
int main() {
  float dist;
  scanf("%f", &dist);
  printf("merters = %f\n", dist*1000);
  printf("feet = %f\n", dist*3.23);
  printf("inches = %f\n", dist*3.23*12);
  printf("centimeters = %f\n", dist*100000);
  return 0;
}
  \end{verbatim} \end{quote} 

  \item \begin{quote} \begin{verbatim}
int main() {
  float s1, s2, s3, s4, s5, ttlMks;
  scanf("%f%f%f%f%f", &s1, &s3, &s3, &s4, &s5);
  ttlMks = s1+ s2+ s3+ s4+ s5;
  printf("aggegrate = %f\n", ttlMks);
  printf("percentage = %f\n", ttlMks*100/500);
  return 0;
}
  \end{verbatim} \end{quote} 

  \item \begin{quote} \begin{verbatim}
int main() {
  float fDeg, cDeg;
  scanf("%f", &fDeg);
  cDeg = (fDeg-32) * 5.0/9;
  printf("centigrade = %f\n", cDeg);
  printf("percentage = %f\n", ttlMks*100/500);
  return 0;
}
  \end{verbatim} \end{quote} 

  \item \begin{quote} \begin{verbatim}
int main() {
  float l, b, r;
  scanf("%f%f%f", &l, &b, &r);
  printf("rectangle perimeter = %f\n", l*b);
  printf("rectangle area = %f\n", 2*(l+b));
  printf("circle circumference = %f\n", 2*3.14*r);
  printf("circle area = %f\n", 3.14*r*r);
  return 0;
}
  \end{verbatim} \end{quote} 

  \item \begin{quote} \begin{verbatim}
int main() {
  int c, d;
  scanf("%d%d", &c, &d);
  c += d;
  d = c - d;
  c = c - d;
  printf("swapped variables c = %d d = %d\n", c, d);
  return 0;
}
  \end{verbatim} \end{quote} 

  \item \begin{quote} \begin{verbatim}
int main() {
  int num, sum = 0;
  scanf("%d", &num);
  sum += num%10;
  num /= 10;
  sum += num%10;
  num /= 10;
  sum += num%10;
  num /= 10;
  sum += num%10;
  num /= 10;
  sum += num;
  printf("sum = %d\n", sum);
  return 0;
}
  \end{verbatim} \end{quote} 
  
  \item \begin{quote} \begin{verbatim}
int main() {
  int num, rev = 0;
  scanf("%d", &num);
  rev = rev*10 + num%10;
  num /= 10;
  rev = rev*10 + num%10;
  num /= 10;
  rev = rev*10 + num%10;
  num /= 10;
  rev = rev*10 + num%10;
  num /= 10;
  rev = rev*10 + num;
  printf("reversed number = %d\n", rev);
  return 0;
}
  \end{verbatim} \end{quote} 

  \item \begin{quote} \begin{verbatim}
int main() {
  int num, sum = 0;
  scanf("%d", &num);
  sum = num%10 + num/1000;
  printf("sum of first & last digit = %d\n", sum);
  return 0;
}
  \end{verbatim} \end{quote} 

  \item \begin{quote} \begin{verbatim}
int main() {
  int menPer = 52, literacy = 48, litMen = 35, pop = 80000;
  int illMen = (menPer-litMen) * pop / 100;
  int illWomen = ((100-53)-(literacy-litMen)) * pop / 100;
  printf("no of illiterate men = %d\n", illMen);
  printf("no of illiterate women = %d\n", illWomen);
  return 0;
}
  \end{verbatim} \end{quote} 

  \item \begin{quote} \begin{verbatim}
int main() {
  int amt;
  scanf("%d", &amt);
  printf("no of 100's = %d\n", amt/100);
  amt %= 100;
  printf("no of 50's = %d\n", amt/50);
  amt %= 50;
  printf("no of 10's = %d\n", amt/10);
  return 0;
}
  \end{verbatim} \end{quote} 

  \item \begin{quote} \begin{verbatim}
int main() {
  int price, profit;
  scanf("%d %d", &price, & profit);
  int cost = (price*15 - profit) / 15;
  printf("cost price per item = %d\n", cost);
  return 0;
}
  \end{verbatim} \end{quote} 

  \item \begin{quote} \begin{verbatim}
int main() {
  int num;
  scanf("%d", &num);
  printf("num after adding 1 to each digit = %d\n", num+11111);
  return 0;
}
  \end{verbatim} \end{quote} 
\end{enumerate}

\chapter{The Decision Control Structure}
\section*{\textit{if, if-else,} Nested \textit{if-elses}}

\textbf{[A] What would be the output of the following programs:}
\begin{enumerate}
    \renewcommand{\labelenumi}{\alph{enumi}}
  \item \begin{quote}
      garabageValue 200\\
  \end{quote}
  \item \begin{quote}
      300 200\\
  \end{quote}
  \item 10 20
  \item \begin{quote}
      3\\
      5\\
  \end{quote}
  \item \begin{quote}
      x and y are equal
  \end{quote}
  \item \begin{quote}
      x = 10 y = 10 z = 0
  \end{quote}
  \item 0 50 0
  \item C is WOW
  \item a = 15 b = 15 c = 0
  \item 1 20 1
\end{enumerate}

\textbf{[B] Point out the errors, if any, in the following programs:}
\begin{enumerate}
    \renewcommand{\labelenumi}{\alph{enumi}}
  \item semantic error: a = b in if condition instead of a == b.
  \item no errors: unnecessary curly brackets are used.
  \item no errors.
  \item syntax error: c doesn't support then keyword.
  \item syntax error: parantheses are necessary for if condition.
  \item syntax error: more than one operand on left of assigment operator
  \item syntax error: elseif in not a valid keyword.
  \item syntax error: c doesn't support then keyword.
  \item semantic error: address of variables must be given in scanf variable, otherwise garbage values will be used.
\end{enumerate}

\textbf{[C] Attempt the following:}
\begin{enumerate}
    \renewcommand{\labelenumi}{\alph{enumi}}
  \item \begin{quote} \begin{verbatim}
int main() {
  int sp, cp;
  scanf("%d %d", &sp, &cp);
  if(cp < sp) {
    printf("profit made = %d\n", sp-cp);
  } else {
    printf("loss made = %d\n", cp-sp);
  }
  return 0;
}
  \end{verbatim} \end{quote} 

  \item \begin{quote} \begin{verbatim}
int main() {
  int num;
  scanf("%d", &num);
  if(num % 2 == 0) {
    printf("even number.\n");
  } else {
    printf("odd number.\n");
  }
  return 0;
}
  \end{verbatim} \end{quote} 

  \item \begin{quote} \begin{verbatim}
int main() {
  int year;
  scanf("%d", &year);
  if(year%400 == 0){
    printf("%d is a leap year!", year);
  } else {
    if(year%4 == 0) {
      if(year%100 == 0) {
        printf("%d is not a leap year!", year);
      } else {
        printf("%d is a leap year!", year);
      }
    } else {
      printf("%d is not a leap year!", year);
    }
  }
  return 0;
}
  \end{verbatim} \end{quote} 

  \item \begin{quote} \begin{verbatim}
int main() {
  int day = 1, month = 13, year;
  scanf("%d", &year);
  year--;
  int k = year % 100;
  int j = year / 100;

  int dayOfWeek = (day + ((13 * (month + 1)) / 5) 
                    + k + (k / 4) + (j / 4) - (2 * j)) % 7;

  if (dayOfWeek < 0) {
    dayOfWeek += 7;
  }

  if(dayOfWeek == 0)
    printf("Saturday \n");
  else if(dayOfWeek == 1)
    printf("Sunday \n");
  else if(dayOfWeek == 2)
    printf("Monday \n");
  else if(dayOfWeek == 3)
    printf("Tuesday \n");
  else if(dayOfWeek == 4)
    printf("Wednesday \n");
  else if(dayOfWeek == 5)
    printf("Thursday \n");
  else
    printf("Friday \n");
  return 0;
}
  \end{verbatim} \end{quote} 

  \item \begin{quote} \begin{verbatim}
int main() {
  int num, rev = 0;
  scanf("%d", &num);
  int temp = num;
  rev = rev*10 + num%10;
  num /= 10;
  rev = rev*10 + num%10;
  num /= 10;
  rev = rev*10 + num%10;
  num /= 10;
  rev = rev*10 + num%10;
  num /= 10;
  rev = rev*10 + num;
  if (temp == rev) {
    printf("%d is equal to %d\n", num, rev);
  } else {
    printf("%d is not equal to %d\n", num, rev);
  }
  return 0;
}
  \end{verbatim} \end{quote} 

  \item \begin{quote} \begin{verbatim}
int main() {
  int ram, shyam, ajay;
  scanf("%d%d%d", &ram, &shyam, &ajay);
  if (ram < shyam) {
    if (ram < ajay) {
      printf("ram is the youngest.");
    } else {
      printf("ajay is the youngest.");
    }
  } else {
    if (shyam < ajay) {
      printf("shyam is the youngest.");
    } else {
      printf("ajay is the youngest.");
    }
  }
  return 0;
}
  \end{verbatim} \end{quote} 

  \item \begin{quote} \begin{verbatim}
int main() {
  int a, b, c;
  scanf("%d%d%d", &a, &b, &c);
  if (a+b+c == 180) {
    printf("is a valid triangle.");
  } else {
    printf("is not a valid triangle.");
  }
  return 0;
}
  \end{verbatim} \end{quote} 

  \item \begin{quote} \begin{verbatim}
int main() {
  int num;
  scanf("%d", &num);
  if (num < 0) {
    printf("absolute value = %d\n", num*(-1));
  } else {
    printf("absolute value = %d\n", num);
  }
  return 0;
}
  \end{verbatim} \end{quote} 

  \item \begin{quote} \begin{verbatim}
int main() {
  int l, b, area, perimeter;
  scanf("%d%d", &l, &b);
  area = l*b;
  perimeter = 2*(l+b);
  if (area < perimeter) {
    printf("area is less than perimeter\n");
  } else {
    printf("area is greater than perimeter\n");
  }
  return 0;
}
  \end{verbatim} \end{quote} 

  \item \begin{quote} \begin{verbatim}
int main() {
  int x1, y1, x2, y2, x3, y3;
  scanf("%d%d%d%d%d%d", &x1, &y1, &x2, &y2, &x3, &y3);
  if ((x2-x2)*(y2-y1) == (x2-x1)*(y3-y2)) {
    printf("points lie on the same line.\n");
  } else {
    printf("points are not collinear.\n");
  }
  return 0;
}
  \end{verbatim} \end{quote} 

  \item \begin{quote} \begin{verbatim}
int main() {
    float cx, cy, radius, px, py;
    printf("coordinates of the center of the circle (x, y): ");
    scanf("%f %f", &cx, &cy);
    printf("radius of the circle: ");
    scanf("%f", &radius);
    printf("Enter the coordinates of the point (x, y): ");
    scanf("%f %f", &px, &py);

    float distance = sqrt(pow(px - cx, 2) + pow(py - cy, 2));

    if (distance < radius) {
        printf("The point is inside the circle.\n");
    } else if (distance == radius) {
        printf("The point is on the circle.\n");
    } else {
        printf("The point is outside the circle.\n");
    }
    return 0;
}
  \end{verbatim} \end{quote} 

  \item \begin{quote} \begin{verbatim}
int main() {
  int x, y;
  scanf("%d%d", &x, &y);
  if (x == 0) {
    if (y == 0) {
      printf("point lie on the origin.\n");
    }
    else {
      printf("point lie on the x axis.\n");
    }
  } else {
    if (x == 0) {
      printf("point lie on the origin.\n");
    }
    else {
      printf("point lie on the y axis.\n");
    }
  }
  return 0;
}
  \end{verbatim} \end{quote} 
\end{enumerate}

\section*{Logical operators}

\textbf{If a = 10, b = 12, c = 0, find the values of the expressions in the following table:}
\begin{quote} \begin{verbatim}
Expression              Value 
a != 6 && b > 5         1
a == 9 || b < 3         0
! ( a < 10 )            1
! ( a > 5 && c )        1 
5 && c != 8 || !c       1
\end{verbatim} \end{quote}

\textbf{[D] What would be the output of the following programs:}
\begin{enumerate}
    \renewcommand{\labelenumi}{\alph{enumi}}
  \item Dean of the students affairs.
  \item Let us C
  \item w = 1 x = 0 y = 1 z = 1
  \item y = 1 z = 1
  \item Bennarivo
  \item 40
  \item Definitely C !
  \item 1 1
  \item z is big
  \item -1 1
  \item k = 0
\end{enumerate}

\textbf{[E] Point out the errors, if any, in the following programs:}
\begin{enumerate}
    \renewcommand{\labelenumi}{\alph{enumi}}
  \item no errors
  \item varibales are not initialized with any values. garbage values will be used.
  \item syntax error: `or' is used instead of `\texttt{||}'.
  \item errors: 
    \subitem in if conditions `=' is used instead of `==' (equality) operator
    \subitem logical operator \&\& misused as it requires boolean values on both side.
  \item syntax error: `and' is used instead of `\&\&'
  \item errors: 
    \subitem variables are not initialized.
    \subitem `\&' is used instead of `\&\&'.
  \item errors:
    \subitem if is closed using `;' causing the else to be orphaned.
  \item error: no errors
\end{enumerate}

\textbf{[F]} Attempt the following: (pg: 89)
\begin{enumerate}
    \renewcommand{\labelenumi}{\alph{enumi}}
  \item \begin{verbatim}
#include <stdio.h>
int main() {
  int year;
  scanf("%d", &year);
  if((year%400 == 0) || (year%4 == 0 && year%100 != 0)) {
    printf("%d is a leap year!", year);
  } else {
    printf("%d is not a leap year!", year);
  }
  return 0;
}
  \end{verbatim}
\item \begin{verbatim}
#include <stdio.h>

int main() {
  char ch;
  scanf("%c", &ch);
  if((ch > -1 && ch < 48) || (ch > 57 && ch < 65) || 
      (ch > 90 && ch < 97) || (ch > 122 && ch < 128)) {
    printf("%c is a special symbol!", ch);
  } else if (ch < 58) {
    printf("%c is a digit!", ch);
  } else if (ch < 91) {
    printf("%c is a capital letter!", ch);
  } else if (ch < 123) {
    printf("%c is a small letter!", ch);
  }
  return 0;
}
  \end{verbatim}
\item \begin{verbatim}
int main() {
  int ensured, age, city, premRate, maxAmt, health, male;

  printf("age: ");
  scanf("%d", &age);
  printf("are you male(0/1): ");
  scanf("%d", &male);
  printf("do you live in city(0/1): ");
  scanf("%d", &city);
  printf("are you in excellent health(0/1):");
  scanf("%d", &health);
  ensured = 1;

  if (health == 1 && (age < 36 && age > 24) && city == 1) {
    if (male == 1) {
      premRate = 4;
      maxAmt = 200000;
    } else {
      premRate = 3;
      maxAmt = 100000;
    }
  } else if (health == 0 && (age < 36 && age > 24) 
              && city == 0 && male == 1) {
      premRate = 6;
      maxAmt = 10000;
  } else {
      ensured = 0;
  }
  if (ensured) {
    printf("You can be ensured for: \n
              premRate: %d && max amount: %d\n"
              , premRate, maxAmt);
  } else {
    printf("You cannot be ensured.\n");
  }
  return 0;
}
  \end{verbatim}
\item \begin{verbatim}
int main() {
  int hardness, tstrength, grade, c1, c2, c3;
  float carbon;
  printf("hardness: ");
  scanf("%d", &hardness);
  printf("carbon content: ");
  scanf("%f", &carbon);
  printf("tensile strength: ");
  scanf("%d", &tstrength);

  c1 = (hardness > 50)? 1: 0;
  c2 = (carbon < 0.7)? 1: 0;
  c3 = (tstrength > 5600)? 1: 0;

  if (c1 && c2 && c3) {
    grade = 10;
  } else if (c1 && c2 && !c3) {
    grade = 9;
  } else if (!c1 && c2 && c3) {
    grade = 8;
  } else if (c1 && !c2 && c3) {
    grade = 7;
  } else if (c1 || c2 || c3) {
    grade = 6;
  } else {
    grade = 5;
  }

  printf("grade = %d\n", grade);

  return 0;
}
  \end{verbatim}
\item \begin{verbatim}
int main() {
  int fine, days;

  printf("enter days: ");
  scanf("%d", &days);

  if (days < 5) {
    printf("fine  50 paise\n");
  } else if (days < 10) {
    printf("fine  1 rupee\n");
  } else if (days < 31) {
    printf("fine  10 rupees\n");
  } else {
    printf("your membership will be cancelled.");
  }

  return 0;
}
  \end{verbatim}
\item \begin{verbatim}
i1nt main() {
  int s1, s2, s3;
  printf("enter the 3 sides of triangle:");
  scanf("%d %d %d", &s1, &s2, &s3);

  if (s1 + s2 > s3 || s1 + s3 > s2 || s3 + s2 > s1) {
    printf("is a valid triangle.\n");
  } else {
    printf("is not a valid triangle.\n");
  }

  return 0;
}
  \end{verbatim}
\item \begin{verbatim}
int main() {
  int s1, s2, s3;
  printf("enter the 3 sides of triangle:");
  scanf("%d %d %d", &s1, &s2, &s3);

  if(s1 == s2 && s2 == s3) {
    printf("is an equilateral triangle.\n");
  } else if (s1 == s2 || s1 == s3 || s2 == s3) {
    printf("is an isosceles triangle.\n");
  } else if (s1*s1 == (s2*s2 + s3*s3) || 
             s2*s2 == (s1*s1 + s3*s3) || 
             s3*s3 == (s1*s1 + s2*s2)) {
    printf("is a right angled triangle.\n");
  } else {
    printf("is an scalene triangle.\n");
  }

  return 0;
}
  \end{verbatim}
\item \begin{verbatim}
int main() {
  int time;
  printf("enter time take to complete the job:");
  scanf("%d", &time);

  if(time >= 2 && time < 3) {
    printf("worker is highly efficient.\n");
  } else if (time >= 3 && time < 4) {
    printf("worker is ordered to increase speed.\n");
  } else if (time >= 4 && time < 5) {
    printf("worker needs to be given training.\n");
  } else if (time > 5) {
    printf("worker is asked to leave company.\n");
  }

  return 0;
}
  \end{verbatim}
\item \begin{verbatim}
int main() {
  int a, b;
  printf("enter percentage obtained in subject A & B:");
  scanf("%d%d", &a, &b);

  if((a >= 55 && b < 45) || (a < 55 && a >= 45 && b >=55)) {
    printf("passed\n");
  } else if (b < 45 && a >= 65) {
    printf("allowed to reapper in B.\n");
  } else {
    printf("Failed.\n");
  }
  return 0;
}
  \end{verbatim}
\item \begin{verbatim}
int main() {
  int order, stock, hasCredit;
  printf("enter values:\norder:");
  scanf("%d", &order);
  printf("stock available:");
  scanf("%d", &stock);
  printf("credit OK (0/1):");
  scanf("%d", &hasCredit);

  if(order <= stock && hasCredit == 1) {
    printf("supply has requirement.\n");
    } else if (hasCredit == 0 && order <= stock) {
    printf("send intimation.\n");
    } else if (hasCredit == 1 && order > stock) {
    printf("send avaliable stock && intimate about balance shipment.\n");
  }
  return 0;
}
  \end{verbatim}
\end{enumerate}

\section*{Conditional operators}
\textbf{[G]} What would be the output of the following programs: (pg: 92)
\begin{enumerate}
    \renewcommand{\labelenumi}{\alph{enumi}}
  \item depends on the garbage value of num:
    \subitem if num < 0; then the output will be 0
    \subitem otherwise; output will be the square of garbage value.
  \item 200
  \item Welcome
\end{enumerate}
\textbf{[H]} Point out the errors, if any, in the following programs: (pg: 93)
\begin{enumerate}
    \renewcommand{\labelenumi}{\alph{enumi}}
  \item syntax error: in conditional statement ? is used instead of :.
  \item syntax error: 2 format specifiers are used but only one variable is passed in printf.
  \item no errors
  \item syntax error in terinary operator, : is not used.
  \item errors:
    \subitem terinary orerator is not complete
    \subitem : is used in front of pfintf func
    \subitem invalid ); is used
  \item no errors
  \item no errors
\end{enumerate}
\textbf{[I]} Rewrite the following programs using conditional operators: (pg: 94)
\begin{enumerate}
    \renewcommand{\labelenumi}{\alph{enumi}}
  \item \begin{verbatim}
  main() {
    int x, min, max;
    scanf("\n%d %d", &max, &x);
    (x > max)? (max = x): (min = x);
  }
  \end{verbatim}
  \item \begin{verbatim}
  main() {
    int code;
    scanf("%d", &code);
    (code > 1)? printf("\nJerusalem"): 
                ((code < 1)? printf("\nEddie"): printf("\nC Brain"));
  }
  \end{verbatim}
  \item \begin{verbatim}
  main() {
    float sal ; 
    printf ("Enter the salary" ) ; 
    scanf ( "%f", &sal ) ; 
    (sal < 40000 && sal > 25000)? printf ( "Manager" ):
        ((sal < 25000 && sal > 15000)? 
        printf ( "Accountant" ): printf ( "Clerk" )); 
  }
  \end{verbatim}
    
\end{enumerate}

\textbf{[J]} Attempt the following: (pg: 95)
\begin{enumerate}
    \renewcommand{\labelenumi}{\alph{enumi}}
  \item \begin{verbatim}
int main() {
  char ch;
  printf("enter character:");
  scanf("%c", &ch);

  ((ch > -1 && ch < 48) || (ch > 57 && ch < 65) ||
   (ch > 90 && ch < 97) || (ch > 122 && ch < 128))
    ? printf("special symbol\n")
    : (ch > 96 && ch < 123)
      ? printf("lower case alphabet\n")
      : printf("some other character\n");

  return 0;
}
  \end{verbatim}
  \item \begin{verbatim}
int main() {
  int year;
  scanf("%d", &year);

  ((year%400 == 0) || (year%4 == 0 && year%100 != 0))
    ? printf("%d is a leap year!", year)
    : printf("%d is not a leap year!", year);
  
  return 0;
}
  \end{verbatim}
  \item \begin{verbatim}
int main() {
  int a, b, c;
  scanf("%d%d%d", &a, &b, &c);

  (a > b && a > c)
    ? printf("%d is the biggest one.\n", a)
    : (b > a && b > c)
      ? printf("%d is the biggest one.\n", b)
      : printf("%d is the biggest one.\n", c);
  
  return 0;
}
  \end{verbatim}
\end{enumerate}



\chapter{The Loop Control Structure}
\section*{\textit{while} loop}

\textbf{[A] What would be the output of the following programs:}
\begin{enumerate}
    \renewcommand{\labelenumi}{\alph{enumi}}
  \item \begin{quote}
      j is not initialized with any value so it will use the garbageValue \\
      already present in it. Making the output uncertain.\\
  \end{quote}

  \item \begin{quote}
  \begin{verbatim}
      
      1
      2
      3
      4
      5
      6
      7
      8
      9
      10
  \end{verbatim}
  \end{quote}

  \item \begin{quote}
      same as part a of this section.\\
  \end{quote}

  \item \begin{quote}
  \begin{verbatim}
      
      0
  \end{verbatim}
  \end{quote}

  \item \begin{quote}
  \begin{verbatim}
      
      0
  \end{verbatim}
  \end{quote}

  \item \begin{quote}
      syntax error in while statement.\\
  \end{quote}

  \item \begin{quote}
  \begin{verbatim}
      
      2 3 3
  \end{verbatim}
  \end{quote}

  \item \begin{quote}
  \begin{verbatim}
      
      3 3 1
  \end{verbatim}
  \end{quote}

  \item \begin{quote}
  \begin{verbatim}
      
      malyalam is a palindrome
      malyalam is a palindrome
      (... infinite loop)
  \end{verbatim}
  \end{quote}

  \item \begin{quote}
  \begin{verbatim}
      
      A computer buff!
      A computer buff!
      (... infinite loop)
  \end{verbatim}
  \end{quote}

  \item \begin{quote}
  \begin{verbatim}
      
      10
      10
      (... infinite loop)
  \end{verbatim}
  \end{quote}

  \item \begin{quote}
  \begin{verbatim}
      
      1.100000
  \end{verbatim}
  \end{quote}

  \item \begin{quote}
  \begin{verbatim}
      
      In while loop
      In while loop
      (... infinite loop)
  \end{verbatim}
  \end{quote}

  \item \begin{quote}
  \begin{verbatim}
      
      Ascii value 0 Character 
      Ascii value 1 Character 
      ...
      ...
      Ascii value 127 Character 
      Ascii value -128 Character 
      Ascii value -127 Character 
      ...
      ...
      Ascii value -1 Character 
      Ascii value 0 Character 
      Ascii value 1 Character 
      ...
      (... infinite loop)
  \end{verbatim}
  \end{quote}

  \item \begin{quote}
  \begin{verbatim}
      
      3 1
      1 3
      0 4
      -1 5
  \end{verbatim}
  \end{quote}

  \item \begin{quote}
  \begin{verbatim}
      
      4 0
      3 1
  \end{verbatim}
  \end{quote}
\end{enumerate}

\textbf{[B] Attempt the following:}
\begin{enumerate}
    \renewcommand{\labelenumi}{\alph{enumi}}
  \item \begin{quote} \begin{verbatim}
int main() {
  int hWorked = 0, overtime = 0;
  printf("enter number of hours worked by employees: ");
  scanf("%d", &hWorked);
  overtime = (hWorked > 40)? 12*(hWorked-40): 0;

  printf("overtime pay = %d\n", overtime);
  return 0;
}
  \end{verbatim} \end{quote}

  \item \begin{quote} \begin{verbatim}
int main() {
  int num = 0, fact = 1;
  printf("enter number: ");
  scanf("%d", &num);
  while(num > 0) {
    fact = fact*num;
    num--;
  }
  printf("factorial = %d\n", fact);
  return 0;
}
  \end{verbatim} \end{quote}

  \item \begin{quote} \begin{verbatim}
int main() {
  int num1 = 0, num2 = 0, res = 1;
  printf("enter 2 numbers: ");
  scanf("%d%d", &num1, &num2);
  while(num2 > 0) {
    res *= num1;
    num2--;
  }
  printf("num1 raised to the num2 = %d\n", res);
  return 0;
}
  \end{verbatim} \end{quote}

  \item \begin{quote} \begin{verbatim}
int main() {
  int x = 0;
  while(x <= 255) {
    printf("Ascii value %d = %c\n", x, x);
    x++;
  }
  return 0;
}
  \end{verbatim} \end{quote}

  \item \begin{quote} \begin{verbatim}
int main() {
  int num = 1, d1, d2, d3;
  while(num <= 500) {
    //the three digits of number 544 = d1d2d3
    d1 = num/100;
    d2 = num/10 % 10;
    d3 = num%10;
    if (d1*d1*d1 + d2*d2*d2 + d3*d3*d3 == num) {
      printf("%d ", num);
    }
    num++;
  }
  return 0;
}
  \end{verbatim} \end{quote}

  \item \begin{quote} \begin{verbatim}
int main() {
  int matchsticks = 21, user;
  int r = 1;
  while(matchsticks > 0) {
    printf("\nRound %d\n", r++);
    printf("your move: \t\t");
    scanf("%d", &user);
    matchsticks -= user;
    if(matchsticks <= 0) {
      printf("\n\nremaining matchsticks = 0\nYOU LOSE!!!\n\n");
      break;
    }
    printf("my move: \t\t%d", 5-user);
    matchsticks -= 5-user;
    printf("\nremaining matchsticks = %d\n", matchsticks);
  }
  return 0;
}
  \end{verbatim} \end{quote}

  \item \begin{quote} \begin{verbatim}
int main() {
  int num, pve, nve, zs, choice;
  pve = nve = zs = 0;
  choice = 1;
  while(choice == 1) {
    printf("\nyour number: ");
    scanf("%d", &num);
    if(num > 0) {
      pve++;
    } else if(num < 0) {
      nve++;
    } else {
      zs++;
    }
    printf("continue?(1/0)\t");
    scanf("%d", &choice);
  }
  printf("numbers entered: \n+ve = %d\n-ve = %d\n0 = %d\n", pve, nve, zs);
  return 0;
}
  \end{verbatim} \end{quote}

  \item \begin{quote} \begin{verbatim}
int main() {
  int num, oct = 0, digits = 1;
  printf("\nyour number: ");
  scanf("%d", &num);
  printf("octal equivalent of %d = ", num);
  while(num > 0) {
    oct = ((num%8) * digits) + oct;
    num /= 8;
    digits *= 10;
  }
  printf("%d\n", oct);
  return 0;
}
  \end{verbatim} \end{quote}

  \item \begin{quote} \begin{verbatim}
 int main() {
  int num, min, max, choice;
  printf("\nyour number: ");
  scanf("%d", &num);
  min = max = num;
  printf("\ncontinue? (1/0) ");
  scanf("%d", &choice);
  while(choice == 1) {
    printf("\nyour number: ");
    scanf("%d", &num);
    if(min > num) {
      min = num;
    } else if(max < num) {
      max = num;
    }
    printf("\ncontinue? (1/0) ");
    scanf("%d", &choice);
  }
  printf("range of entered numbers = %d\n", max-min);
  return 0;
} \end{verbatim} \end{quote}
\end{enumerate}



\section*{\textit{for, break, continue, do-while}}

\textbf{[C] What would be the output of the following programs:}
\begin{enumerate}
    \renewcommand{\labelenumi}{\alph{enumi}}
  \item \begin{quote}
      no output\\
  \end{quote}

  \item \begin{quote} \begin{verbatim}

2
3
4
5
6
  \end{verbatim} \end{quote}

  \item \begin{quote} \begin{verbatim} 

2
5
  \end{verbatim} \end{quote}

  \item \begin{quote} \begin{verbatim} 

A
A
A
A
A
  \end{verbatim} \end{quote}
\end{enumerate}

\textbf{[D] Answer the following:}
\begin{enumerate}
    \renewcommand{\labelenumi}{\alph{enumi}}
  \item \begin{quote}
      initialize loop counter\\
      test\\
      incrementing/decrementing counter\\
  \end{quote}

  \item \begin{quote}
      arithmetic, relational, assignment\\
  \end{quote}

  \item \begin{quote}
      a for loop\\
  \end{quote}

  \item \begin{quote}
      at least once\\
  \end{quote}

  \item \begin{quote}
      initialization, execution of body, testing\\
  \end{quote}

  \item \begin{quote}
      3 is not an infinite loop\\
  \end{quote}

  \item \begin{quote}
      continue\\
  \end{quote}
\end{enumerate}


\textbf{[E] Attempt the following:}
\begin{enumerate}
    \renewcommand{\labelenumi}{\alph{enumi}}
  \item \begin{quote} \begin{verbatim} 
#include <stdio.h>
#include <math.h>

int main() {
  int num, i, sr, isPrime;
  for(num = 1; num <= 300; num++) {
    isPrime = 1;
    i = 2;
    sr = sqrt(num);
    for(i = 2; i <= sr; i++) {
      if(num%i == 0) {
        isPrime = 0;
        break;
      }
    }
    if(isPrime) {
      printf("%d ", num);
    }
  }
  return 0;
}
  \end{verbatim} \end{quote}

  \item \begin{quote} \begin{verbatim} 
int main() {
  int i;
  for(i = 1; i <= 30000; i++) {
    printf("%c", 1);
    \\ printf("%s", "\u263a");
  }
  return 0;
}
  \end{verbatim} \end{quote}

  \item \begin{quote} \begin{verbatim} 
int main() {
  int i, num;
  float sum = 0, fact;
  for(i = 1; i <= 7; i++) {
    num = i;
    fact = 1;
    while(num > 0) {
      fact = fact*num;
      num--;
    }
    sum += i/fact;
  }
  printf("sum of first 7 terms = %f\n", sum);
  return 0;
}
  \end{verbatim} \end{quote}

  \item \begin{quote} \begin{verbatim} 
int main() {
  int i, j, k;
  for(i = 1; i <= 3; i++) {
    for(j = 1; j <= 3; j++) {
      for(k = 1; k <= 3; k++) {
        printf("%d%d%d\n", i, j, k);
      }
    }
  }
  return 0;
}
  \end{verbatim} \end{quote}

  \item \begin{quote} \begin{verbatim} 
int main() {
  float i, x;
  int y;
  printf("\t\t\t1\t\t\t2\t\t\t3\t\t\t4\t\t\t5\t\t\t6");
  for(x = 5.5; x <= 12.5; x += 0.5) {
    printf("\n%4.1f\t", x);
    for(y = 1; y <= 6; y++) {
      i = 2 + (y + (0.5 * x));
      printf("%.2f\t", i);
    }
  }
  return 0;
}
  \end{verbatim} \end{quote}

  \item \begin{quote} \begin{verbatim} 
int main() {
  int i, j;
  for(i = 0; i < 7; i++) {
    for(j = 0; j < 13; j++) {
      if((7-i <= j && j < 7) || (i-1 > j%7 && j/7)) {
        printf("  ");
        continue;
      }
      if(j < 7) {
        printf("%c ", 'A' + j);
      } else {
        printf("%c ", 'F' - j%7);
      }
    }
    printf("\n");
  }
  return 0;
}
  \end{verbatim} \end{quote}

  \item \begin{quote} \begin{verbatim} 
int main() {
  int i;
  for(i = 0; i < 30000; i++) {
    //printf("%c%c", 3, 4);
    printf("%s%s", "\u2661", "\u2662");
  }
  return 0;
}
  \end{verbatim} \end{quote}

  \item \begin{quote} \begin{verbatim} 
int main() {
  int i, num;
  scanf("%d", &num);
  for(i = 1; i <= 10; i++) {
    printf("%d * %2d = %3d\n", num, i, num*i);
  }
  return 0;
}
  \end{verbatim} \end{quote}

  \item \begin{quote} \begin{verbatim} 
int main() {
  int i, j, num = 0;
  for(i = 0; i < 4; i++) {
    for(j = 0; j < 4; j++) {
      if(i+j < 3) {
        printf(" ");
      } else {
        printf("%d ", ++num);
      }
    }
    printf("\n");
  }
  return 0;
}
  \end{verbatim} \end{quote}

  \item \begin{quote} \begin{verbatim} 
int main() {
  int i, j, n, nmr, r, nfact, nmrfact, rfact;
  for(i = 0; i < 5; i++) {
    for(j = 5-i; j > 0; j--) {
      printf(" ");
    }
    for(j = 0; j <= i; j++) {
        n = i; r = j; nmr = n-r;
        nfact = nmrfact = rfact = 1;
        while(n > 0) {
          nfact *= n;
          n--;
        }
        while(nmr > 0) {
          nmrfact *= nmr;
          nmr--;
        }
        while(r > 0) {
          rfact *= r;
          r--;
        }
        printf("%d ", nfact / (nmrfact*rfact));
    }
    printf("\n");
  }
  return 0;
}
  \end{verbatim} \end{quote}

  \item \begin{quote} \begin{verbatim} 
int main() {
  int mCost = 6000, mEarning = 1000, mSalvage = 2000;
  int year = 0;
  float altEarn = 0, mEarn = 0, iRate = 12;
  while(altEarn >= mEarn) {
    altEarn += (altEarn + 4000) * iRate / 100;
    mEarn += 1000;
    year++;
  }
  printf("minimum life = %d\n", year);

  return 0;
}
  \end{verbatim} \end{quote}

  \item \begin{quote} \begin{verbatim} 
int main() {
  float p, r, n, q, nq; 
  double amount = 0, exp, expNq;
  int i = 0, j;
  while(i < 10) {
    scanf("%f%f%f%f", &p, &r, &n, &q);
    j = 0; 
    exp = 1 + r/q;
    expNq = 1;
    nq = n*q;
    while(j < nq) {
      expNq *= exp;
      j++;
    }
    amount = p * expNq;
    printf("amount = %lf\n", amount);
    i++;
  }

  return 0;
}
  \end{verbatim} \end{quote}

  \item \begin{quote} \begin{verbatim} 
int main() {
  float x; 
  scanf("%f", &x);
  float exp, expn, nlog = (x-1)/x;
  expn =  exp = nlog;

  int i = 2;
  while(i < 8) {
    expn *= exp;
    nlog += expn/2;
    i++;
  }
  printf("natural log of first terms = %f\n", nlog);

  return 0;
}
  \end{verbatim} \end{quote}
\end{enumerate}





\chapter{The Case Control Structure}

\textbf{[A] What would be the output of the following programs:}
\begin{enumerate}
    \renewcommand{\labelenumi}{\alph{enumi}}
  \item \begin{quote} \begin{verbatim} 

Heart
I thought one wears a suite
  \end{verbatim} \end{quote}

  \item \begin{quote} \begin{verbatim} 
I am in case 3
  \end{verbatim} \end{quote}

  \item \begin{quote} \begin{verbatim} 

Pure Simple Egghead!
  \end{verbatim} \end{quote}

  \item \begin{quote} \begin{verbatim} 

Customers are dicey
Markets are pricey
Inverstors are moody
At least employees are good
  \end{verbatim} \end{quote}

  \item \begin{quote} \begin{verbatim} 

Trapped
  \end{verbatim} \end{quote}

  \item \begin{quote} \begin{verbatim} 

You entered a and b
  \end{verbatim} \end{quote}

  \item \begin{quote} \begin{verbatim} 

Feeding fish
Weeking grass
mending roof
Just to survive
  \end{verbatim} \end{quote}
\end{enumerate}


\textbf{[B] Point out the errors, if any, in the following programs:}
\begin{enumerate}
    \renewcommand{\labelenumi}{\alph{enumi}}
  \item \begin{quote} \begin{verbatim} 
syntax error in case 0 & 1; 
Also case statements are not allowed outside switch statement.
  \end{verbatim} \end{quote}

  \item \begin{quote} \begin{verbatim} 
error: expression in case is not integer constant. (operand is not a constant)
  \end{verbatim} \end{quote}

  \item \begin{quote} \begin{verbatim} 
error: quantity in switch is not an integer.
  \end{verbatim} \end{quote}

  \item \begin{quote} \begin{verbatim} 
error: 2nd case statement is not an integer constant,
variables a and b are not considered constatns.
  \end{verbatim} \end{quote}
\end{enumerate}


\textbf{[C] Write a menu driven program which has following options:}
\begin{enumerate}
  \item Factorial of a number.
  \item Prime or not
  \item Odd or even
  \item Exit
  \begin{verbatim} 
#include <stdio.h>
#include <math.h>

int main() {
  int choice, i, sr, num, fact;
  while(1) {
    printf("\n1. Factorial");
    printf("\n2. Prime");
    printf("\n3. Odd/Even");
    printf("\n4. Exit");
    printf("\nYour choice?");
    scanf("%d", &choice);

    switch (choice) {
      case 1:
        printf("\nenter number: ");
        scanf("%d", &num);
        fact = 1;
        while(num > 0) {
          fact = fact*num;
          num--;
        }
        printf("factorial = %d\n", fact);
        break;
      case 2:
        printf("\nenter number: ");
        scanf("%d", &num);
        i = 2;
        sr = sqrt(num);
        for(i = 2; i <= sr; i++) {
          if(num%i == 0) {
            printf("%d is a prime number.\n", num);
            break;
          }
        }
        break;
      case 3:
        printf("\nenter number: ");
        scanf("%d", &num);
        if(num%2 == 0) {
          printf("%d is an even number.\n", num);
        } else {
          printf("%d is an odd number.\n", num);
        }
        break;
      case 4:
        return 0;
    }
  }
  
  return 0;
}
  \end{verbatim} 
\end{enumerate}

\textbf{[D] Write a program to find the grace marks for a student using switch.
The user should enter the class obtained by the student and the number of 
subjects he has failed in.:}
\begin{verbatim} 
#include <stdio.h>

int main() {
  int class, noOfSubs, grace = 0;
  printf("enter class obtained by student: ");
  scanf("%d", &class);
  printf("number of subjects failed in: ");
  scanf("%d", &noOfSubs);

  switch (class) {
    case 1:
      if(noOfSubs <= 3) {
        grace += noOfSubs * 5;
      }
      break;
    case 2:
      if(noOfSubs <= 2) {
        grace += noOfSubs * 4;
      }
      break;
    case 3:
      if(noOfSubs <= 1) {
        grace += noOfSubs * 5;
      }
      break;
  }

  printf("grace marks for student = %d\n", grace);
  
  return 0;
}
  \end{verbatim} 



\end{document}
